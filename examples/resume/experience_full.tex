%-------------------------------------------------------------------------------
%	SECTION TITLE
%-------------------------------------------------------------------------------
\cvsection{Work Experience}


%-------------------------------------------------------------------------------
%	CONTENT
%-------------------------------------------------------------------------------
\begin{cventries}

%---------------------------------------------------------
\cventry
{Senior Consultant} % Job title
{Quantitative Trading Book in Ernst \& Young U.S. LLP} % Organization
{New York, USA} % Location
{Oct 2023 - Present} % Date(s)
{
  \begin{cvitems}
    \item \textbf{Assistant AI System for Intelligent Data Retrieval and Analysis}
    \begin{itemize}
      \item Developed an AI assistant that interprets user input and retrieves relevant data from internal databases for automated analysis and summarization.
      \item Implemented a \textbf{Retrieval-Augmented Generation (RAG)} architecture: embedded structured and unstructured content—including text, tables, and images—using multi-modal encoders.
      \item Transformed user queries into dense vector representations and performed semantic search to identify top-matching data entries.
      \item Constructed dynamic prompts from retrieved content to interface with \textbf{Google Gemini} and other LLMs, enabling context-aware, analytical, and explainable AI-generated responses.
    \end{itemize}
    \item \textbf{Automated Sensitive Information Detection and Classification}
    \begin{itemize}
      \item Built a neural network system to automatically assess and classify documents based on the presence of sensitive information (e.g., customer records, internal data) at the moment of file creation or saving.
      \item Embedded document content into vector representations, capturing semantic patterns across text and metadata.
      \item Designed a shared neural architecture: applied a shared encoder layer followed by task-specific classification heads for each sensitivity category.
      \item Combined multi-head outputs to determine the overall confidentiality level, enabling real-time access control and compliance labeling.
    \end{itemize}
    \item \textbf{Applied Machine Learning Projects in Generative AI Context}
    \begin{itemize}
      \item \textbf{Item Blurring Pipeline}: Built a two-stage object detection and classification framework with a Region Proposal Network and classifier to automatically blur specified items in images.
      \item \textbf{Harmful Content Detection}: Designed a multi-task classification model with early fusion of text and image features to detect potentially harmful or policy-violating email content.
      \item \textbf{Ad Click Prediction}: Constructed a personalized advertising model using Gradient Boosting Decision Trees (\textbf{GBDT}) and \textbf{DeepFM}, improving ad relevance and user engagement.
    \end{itemize}
    \item Modular Redesign of Derivatives Pricing Algorithm
    \begin{itemize}
      \item Led the architectural overhaul by decomposing the algorithm into service class and analysis units, archieving high \textbf{decoupling} of code.
      \item Enabling independent updates to each component without affecting the overall system, significantly reducing redundancy and enhancing maintainability.
      \item Designed robust unit testing frameworks, improving system \textbf{debug reliability} by proactively identifying potential errors.
    \end{itemize}
    \item Optimization of American Options Pricing
    \begin{itemize}
      \item Applied the American Monte Carlo (\textbf{AMC}) method to price American options, replacing the original Monte Carlo over Monte Carlo method. 
      \item Achieved a substantial reduction in computational complexity from O(n²) to \textbf{O(n)}, cutting pricing time and saving considerable resources.
    \end{itemize}
    \item Equity Derivatives Pricing Algorithm Enhancement
    \begin{itemize}
      \item Improved the pricing framework for equity derivatives by transitioning from a market-based risk model to an underlying location-based risk analysis, enhancing accuracy and \textbf{interpretablity}.
      \item Intergrated advanced machine learning techniques, such as \textbf{LSTM}, \textbf{random forest} models with traditional MCMC methods to price derivatives, enabling the pricing of complex toxic options with more than three underlying.
    \end{itemize}
    \item Counterparty Credit Risk Monitoring
    \begin{itemize}
      \item Employed SFT VaR-based models to calculate and monitor Counterparty Credit Risk.
      \item \textbf{Interpreted} complex data and model results, and delivered clear insights to stakeholders, including cross-disciplinary teams and \textbf{non-technical} audiences.
      \item Regularly updated model parameters in line with evolving market data, ensuring the models reflect current market conditions and deliver accurate risk assessments.
    \end{itemize}
  \end{cvitems}
}


%---------------------------------------------------------
  \cventry
    {Securities Analyst Assistant (intern)} % Job title
    {Bank of China International Holdings Limited} % Organization
    {Shanghai, China} % Location
    {Jun 2021-Sep 2021} % Date(s)
    {
      \begin{cvitems} % Description(s) of tasks/responsibilities
        \item {Focused on battery and new energy industry. Predicted the short- and long-term performance of stocks of related companies based on time series model with a spike-and-slab error.}
        \item {Adjusted the prediction under a multinomial model based on the performance of correlated companies and avoided making an over-optimistic forecast compared with previous model.}
      \end{cvitems}
    }

%---------------------------------------------------------
  \cventry
    {Data Analyst (intern)} % Job title
    {HUATAI SECURITIES CO., LTD.(HTSC)} % Organization
    {Jiangsu, China} % Location
    {Jul 2017-Sep 2017} % Date(s)
    {
      \begin{cvitems} % Description(s) of tasks/responsibilities
        \item {Unsupervised screened visitors with a strong desire to buy products based on their records on company’s APP.}
        \item {Cleaned and reshaped the 17 million visitor records by summarizing operations from the same visitor.}
        \item {Extracted useful variables by PCA (principal component analysis) method.}
        \item {Divided visitors into five groups by K-means methods and assigned visitors labels by their group.}
        \item {Fitted a decision tree with labeled data which could tag new visitor within 20 seconds while the target is 1 min.}
      \end{cvitems}
    }


  
%---------------------------------------------------------
  \cventry
    {Project Leader} % Job title
    {Statistical Consulting Group of University of Connecticut} % Organization
    {Connecticut, USA} % Location
    {Sep 2020 - Sep 2023} % Date(s)
    {
      \begin{cvitems}
        \item Credit Card Approval with Unbalanced Data and Outliers 
        \begin{itemize}
          \item Decide who to approve or decline for credit based on historical repayment records.
          \item Adding new missing indicator variables before applying imputing missing value after checking randomness. 
          \item Generate features based on the distribution of outliers and assign different weights on unbalanced responses.
          \item Fit logistic regression, XGBoost, and Random Forest models separately and use the linear combination of three models as final model after cross validation.
        \end{itemize}
        \item Yelp Reviews Rating Prediction
        \begin{itemize}
          \item Predicted Yelp reviews’ rating on 1 million unlabeled text reviews.
          \item Cleaned 1.5 million Yelp reviews by removing un-English comments, abbreviations, and spelling mistakes. 
          \item Extracted positive/negative words based on their relative frequency in differently rated reviews to avoid placing too much weight on everyday words like “the”, “a” which can be mistaken as positive words.
          \item	Transfer text reviews into vectors by Sentence-To-Vector and generate new features from positive/negative words.
          \item Fitted pre-processed data by Long-Short-Term-Memory (LSTM) neural network and achieved 0.6 root-mean-square-error.
        \end{itemize}
      \end{cvitems}
    }


%---------------------------------------------------------
\end{cventries}
