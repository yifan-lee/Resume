%-------------------------------------------------------------------------------
%	SECTION TITLE
%-------------------------------------------------------------------------------
\cvsection{Work Experience}


%-------------------------------------------------------------------------------
%	CONTENT
%-------------------------------------------------------------------------------
\begin{cventries}

%---------------------------------------------------------
  \cventry
    {Senior Consultant} % Job title
    {Quantitative Trading Book in Ernst \& Young U.S. LLP} % Organization
    {New York, USA} % Location
    {Oct 2023 - Present} % Date(s)
    {
      \begin{cvitems}
        \item Modular Redesign of Derivatives Pricing Algorithm
        \begin{itemize}
          \item Led the architectural overhaul by decomposing the algorithm into service class and analysis units, archieving high \textbf{decoupling} of code.
          \item Enabling independent updates to each component without affecting the overall system, significantly reducing redundancy and enhancing maintainability.
          \item Designed robust unit testing frameworks, improving system \textbf{debug reliability} by proactively identifying potential errors.
        \end{itemize}
        \item Optimization of American Options Pricing
        \begin{itemize}
          \item Applied the American Monte Carlo (\textbf{AMC}) method to price American options, replacing the original Monte Carlo over Monte Carlo method. 
          \item Achieved a substantial reduction in computational complexity from O(n²) to \textbf{O(n)}, cutting pricing time and saving considerable resources.
        \end{itemize}
        \item Equity Derivatives Pricing Algorithm Enhancement
        \begin{itemize}
          \item Improved the pricing framework for equity derivatives by transitioning from a market-based risk model to an underlying location-based risk analysis, enhancing accuracy and \textbf{interpretablity}.
          \item Intergrated advanced machine learning techniques, such as \textbf{LSTM}, \textbf{random forest} models with traditional MCMC methods to price derivatives, enabling the pricing of complex toxic options with more than three underlying.
        \end{itemize}
        \item Counterparty Credit Risk Monitoring
        \begin{itemize}
          \item Employed SFT VaR-based models to calculate and monitor Counterparty Credit Risk.
          \item \textbf{Interpreted} complex data and model results, and delivered clear insights to stakeholders, including cross-disciplinary teams and \textbf{non-technical} audiences.
          \item Regularly updated model parameters in line with evolving market data, ensuring the models reflect current market conditions and deliver accurate risk assessments.
        \end{itemize}
      \end{cvitems}
    }


%---------------------------------------------------------
  \cventry
    {Securities Analyst Assistant (intern)} % Job title
    {Bank of China International Holdings Limited} % Organization
    {Shanghai, China} % Location
    {Jun 2021-Sep 2021} % Date(s)
    {
      \begin{cvitems} % Description(s) of tasks/responsibilities
        \item {Predicted the short- and long-term performance of new energy industry equity based on time series model with a spike-and-slab error.}
        \item {Adjusted the prediction under a multinomial model based on the performance of correlated companies, avoiding an over-optimistic forecast.}
      \end{cvitems}
    }

%---------------------------------------------------------
  \cventry
    {Data Analyst (intern)} % Job title
    {HUATAI SECURITIES CO., LTD. (HTSC)} % Organization
    {Jiangsu, China} % Location
    {Jul 2017-Sep 2017} % Date(s)
    {
      \begin{cvitems} % Description(s) of tasks/responsibilities
        \item {\textbf{Unsupervised} screened visitors with a strong desire to buy products based on their records on company’s APP.}
        \item {Cleaned and reshaped the \textbf{17 million} visitor records by summarizing operations from the same visitor, and grouped them by \textbf{K-means}.}
        \item {Extracted useful variables by principal component analysis (\textbf{PCA}) method used in decision tree to tag visitor in 20s while the target is 1 min.}
      \end{cvitems}
    }



%---------------------------------------------------------
\end{cventries}
